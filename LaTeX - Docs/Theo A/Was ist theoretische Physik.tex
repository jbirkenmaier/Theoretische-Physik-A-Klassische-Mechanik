Theoretische Physik ist die Anwendung mathematischer Werkzeuge zur Beschreibung von Naturphänomenen. Die auftretenden Phänomene sind im Allgemeinen dynamisch, das bedeutet sie ändern sich in Abhängigkeit von der Zeit oder von dem Ort. \newline
Es gibt einige fundamentale Gleichungen, in welchen alle Gesetze der Physik reflektiert werden.  
Sie enthalten die Gesetze, welche uns Vorhersagen über das Verhalten von Systemen machen lassen. 
Diese Gleichungen haben die Form von sogenannten Differentialgleichungen. Das sind Gleichungen, die beschreiben wie sich bestimmte Größen in Abhänigkeit von anderen Größen in Abhängigkeit von Parametern ändern. 
Ein Beispiel hierfür ist die Newton'sche Bewegungsgleichung
\begin{align}
	m
\end{align}


Wichtige Differentialgleichungen sind zum Beispiel die Newtonsche Bewegungsgleichung, die Euler-Lagrange-Gleichung, die Maxwellgleichungen und die Schrödingergleichung.

