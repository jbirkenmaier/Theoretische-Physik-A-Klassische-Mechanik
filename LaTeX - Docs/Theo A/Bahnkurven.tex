\section{Bahnkurven}
\subsection{Der Massepunkt}
Häufig kann die räumliche Ausdehnung eines Körpers vernachlässigt werden. Die gesamte Masse dieses Körpers wird dann in einem einzelnen Punkt im Raum angenommen. Dieser Ort kann mithilfe eines Vektors angegeben werden, dem sogenannten Ortsvektor.

\subsection{Der Ortsvektor}
Der Ortsvektor ist ein mathematisches Objekt, das die Position eines Punktes im Raum festlegt. Orte müssen im Allgemeinen nicht in kartesischen Koordinaten angegeben werden. Ein grundsätzliches Prinzip in der theoretischen Physik ist es, stets das Koordinatensystem zu wählen, welches das Problem am stärksten vereinfacht. Zu Koordinatensystemen später mehr. 
Konventionell wird der Ortsvektor mit dem Buchstaben $\vec{r}$ benannt.

\begin{align}
	\vec{r} = \cvec{x_1\\x_2\\x_3} 
\end{align}

\subsection{Die Geschwindigkeit}
Meistens sind wir nicht nur an dem Ort eines Objekts interessiert sondern auch an dessen Geschwindigkeit. Denn allein durch eine Ortsangabe können wir noch keine Aussage darüber machen, wie das mechanische System sich in der Zukunft verhalten wird. Eine reine Ortsangabe ist sozusagen bloß der Schnappschuss eines mechanischen Systems. 
Wir würden unser System gerne fragen, wie es sich in der Zukunft entwickeln wird. Solche Fragen an Systeme werden in der Physik mit sogenannten Operatoren formuliert. Ein sehr einfacher Operator ist die zeitliche Ableitung. 
Wenn wir also Information darüber erhalten wollen, wie ein Ortsvektor sich in der Zeit entwickelt, so ist die zeitliche Ableitung das Mittel der Wahl zur Formulierung dieser Frage. 
\begin{align}
	\ddt{\vec{r}} = \text{Änderung des Orts mit der Zeit}
\end{align}

Die Änderung des Ortes mit der Zeit wird \textit{Geschwindigkeit} $\vec{v}$ genannt.

\begin{align}
	\vec{v} = \text{Geschwindigkeit}
\end{align}

Zur Ableitung eines Vektors wird jede Komponente einzeln abgeleitet.

\begin{align}
	\vec{v}=\ddt{\vec{r}} = \cvec{\ddt{x_1}\\\ddt{x_2}\\\ddt{x_3}}
\end{align}

Eine vereinfachte Schreibweise für Ableitungen ist es,  einen Punkt über der abgeleiteten Größe zu schreiben
\begin{align}
	\ddt{\vec{r}} \equiv \dot{\vec{r}}
\end{align}

\subsection{Die Beschleunigung}
Häufig ist man daran interessiert, wie sich die Geschwindigkeit mit der Zeit entwickelt. Erneut wirkt die zeitliche Ableitung als Frage und liefert die Antwort, die wir als Beschleunigung $\vec{a}$ bezeichnen.

\begin{align}
	\vec{a}=\dot{\vec{v}}=\dsqdt{\vec{r}} = \cvec{\dsqdt{x_1}\\\dsqdt{x_2}\\\dsqdt{x_3}}
\end{align}

\subsection{Freiheitsgrade}
Die Anzahl der Freiheitsgrade eines Systems ist die Anzahl der nötigen Größen um die \textit{Lage} eines Systems zu beschreiben. Anders gesagt, es ist die Anzahl nötiger Größen um zu beschreiben wo alle relevanten Objekte im Raum liegen. Ein einzelner Massenpunkt hat also in einem dreidimensionalen Raum drei Freiheitsgrade. Ein System mit N Massenpunkten hat somit $3N$ Freiheitsgrade.

\subsection{Beispiele}
pass


