\section{Koordinatensysteme}

Koordinatensysteme werden zur Beschreibung physikalischer Systeme verwendet. Allgemein bekannt ist das kartesische Koordinatensystem.
Eigentlich benötigt man nur drei verschiedene Arten von Koordinatensystemen, das kartesische Koordinatensystem,  das Zylinderkoordinatensystem und das Kugelkoordinatensystem. 

\begin{tikzpicture}
    % Draw axes
    \draw[->] (0,0) -- (3,0) node[right] {$x$}; % x-axis
    \draw[->] (0,0) -- (0,3) node[above] {$y$}; % y-axis
    \draw[->] (0,0) -- (-2,-2) node[below] {$z$}; %z-axis

    % Add grid (optional)
    %\draw[very thin, gray] (-3,-3) grid (3,3);

    % Add labels for the axes
    %\foreach \x in {-2,-1,1,2} % x-axis labels
    %    \draw (\x,0.1) -- (\x,-0.1) node[below] {\x};
    %\foreach \y in {-2,-1,1,2} % y-axis labels
    %    \draw (0.1,\y) -- (-0.1,\y) node[left] {\y};
        
    % Example point
    \filldraw[red] (1,2) circle (2pt) node[above] {$(x,y,z)$};
\end{tikzpicture}

\begin{tikzpicture}
    % Draw axes
    \draw[->] (0,0) -- (3,0) node[right] {$y$}; 
    \draw[->] (0,0) -- (0,3) node[above] {$z$}; 
    \draw[->] (0,0) -- (-2,-2) node[below] {$x$}; 

    % Draw spherical point
    \coordinate (O) at (0,0); % Origin
    \coordinate (P) at (1.5,1); % Point in spherical coordinates
    \coordinate (ProjP) at (1.5,-2); % Projection of P onto the x-y plane

    \draw[->, thick, red] (O) -- (1.4,0.93) node[above] {$\vec{ r }$}; % Radius vector
    \draw[dashed, gray] (P) -- (ProjP) node[below right] {}; % Projection line
    \draw[dashed, gray] (O) -- (ProjP) node[below] {}; % Projection vector in x-y plane

    % Add theta (polar angle)
    \draw[->, thick, blue] (0.5,0.33) arc[start angle=10, end angle=70, radius=0.7cm];
    \node[blue] at (0.2,0.5) {$\theta$};

    % Add phi (azimuthal angle)
    \draw[->, thick, green] (-0.5,-0.5) arc[start angle=-135, end angle=-40, radius=0.55cm];
    \node[green] at (-0.05,-0.38) {$\varphi$};

      % Draw labels
%    \node[red] at (1.5,1.5) {$(r, \theta, \phi)$};
    \filldraw[red] (P) circle (2pt); % Spherical point
\end{tikzpicture}

\begin{tikzpicture}
    % Draw axes
    \draw[->] (0,0) -- (3,0) node[right] {$y$}; 
    \draw[->] (0,0) -- (0,3) node[above] {$z$}; 
    \draw[->] (0,0) -- (-2,-2) node[below] {$x$}; 

    % Draw cylindrical point
    \coordinate (O) at (0,0); % Origin
    \coordinate (P) at (1.5,2); % Point in cylindrical coordinates
    \coordinate (ProjP) at (1.5,0); % Projection of P onto the x-y plane

    % Draw radial vector
    \draw[->, thick, red] (O) -- (1.5,0) node[below right] {$r$}; % Radial distance
    \draw[->, thick, gray] (1.5,0) -- (P) node[right] {$z$}; % Height

    % Add dashed line for projection
    \draw[dashed, gray] (O) -- (P) node[above right] {}; % Radius to point

    % Add phi (azimuthal angle)
    \draw[->, thick, blue] (0.7,0) arc[start angle=0, end angle=45, radius=0.7cm];
    \node[blue] at (0.8,0.3) {$\varphi$};

    % Draw labels
    \filldraw[red] (P) circle (2pt) node[above] {$(r, \varphi, z)$}; % Cylindrical point
\end{tikzpicture}