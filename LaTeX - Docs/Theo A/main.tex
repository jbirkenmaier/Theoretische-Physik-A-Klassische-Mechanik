\documentclass[a4paper, 12pt]{book}

% Sprachunterstützung
%\usepackage[ngerman]{babel} % Für deutsche Sprache
%\usepackage[utf8]{inputenc} % UTF-8 Eingabe
%\usepackage[T1]{fontenc}    % Korrekte Trennung für Umlaute
\usepackage{polyglossia}
\setdefaultlanguage[spelling=new]{german}
\usepackage{tikz}

\usepackage{amsmath}

\newcommand{\cvec}[1]{\begin{pmatrix} #1 \end{pmatrix}}
\newcommand{\ddt}[1]{\frac{\text{d}}{\text{d}t}\left[#1\right]}
\newcommand{\dsqdt}[1]{\frac{\text{d}^2}{\text{d}t^2}\left[#1\right]}

\begin{document}

\tableofcontents

\chapter{Kinematik}
\section{Bahnkurven}
\subsection{Der Massepunkt}
Häufig kann die räumliche Ausdehnung eines Körpers vernachlässigt werden. Die gesamte Masse dieses Körpers wird dann in einem einzelnen Punkt im Raum angenommen. Dieser Ort kann mithilfe eines Vektors angegeben werden, dem sogenannten Ortsvektor.

\subsection{Der Ortsvektor}
Der Ortsvektor ist ein mathematisches Objekt, das die Position eines Punktes im Raum festlegt. Orte müssen im Allgemeinen nicht in kartesischen Koordinaten angegeben werden. Ein grundsätzliches Prinzip in der theoretischen Physik ist es, stets das Koordinatensystem zu wählen, welches das Problem am stärksten vereinfacht. Zu Koordinatensystemen später mehr. 
Konventionell wird der Ortsvektor mit dem Buchstaben $\vec{r}$ benannt.

\begin{align}
	\vec{r} = \cvec{x_1\\x_2\\x_3} 
\end{align}

\subsection{Die Geschwindigkeit}
Meistens sind wir nicht nur an dem Ort eines Objekts interessiert sondern auch an dessen Geschwindigkeit. Denn allein durch eine Ortsangabe können wir noch keine Aussage darüber machen, wie das mechanische System sich in der Zukunft verhalten wird. Eine reine Ortsangabe ist sozusagen bloß der Schnappschuss eines mechanischen Systems. 
Wir würden unser System gerne fragen, wie es sich in der Zukunft entwickeln wird. Solche Fragen an Systeme werden in der Physik mit sogenannten Operatoren formuliert. Ein sehr einfacher Operator ist die zeitliche Ableitung. 
Wenn wir also Information darüber erhalten wollen, wie ein Ortsvektor sich in der Zeit entwickelt, so ist die zeitliche Ableitung das Mittel der Wahl zur Formulierung dieser Frage. 
\begin{align}
	\ddt{\vec{r}} = \text{Änderung des Orts mit der Zeit}
\end{align}

Die Änderung des Ortes mit der Zeit wird \textit{Geschwindigkeit} $\vec{v}$ genannt.

\begin{align}
	\vec{v} = \text{Geschwindigkeit}
\end{align}

Zur Ableitung eines Vektors wird jede Komponente einzeln abgeleitet.

\begin{align}
	\vec{v}=\ddt{\vec{r}} = \cvec{\ddt{x_1}\\\ddt{x_2}\\\ddt{x_3}}
\end{align}

Eine vereinfachte Schreibweise für Ableitungen ist es,  einen Punkt über der abgeleiteten Größe zu schreiben
\begin{align}
	\ddt{\vec{r}} \equiv \dot{\vec{r}}
\end{align}

\subsection{Die Beschleunigung}
Häufig ist man daran interessiert, wie sich die Geschwindigkeit mit der Zeit entwickelt. Erneut wirkt die zeitliche Ableitung als Frage und liefert die Antwort, die wir als Beschleunigung $\vec{a}$ bezeichnen.

\begin{align}
	\vec{a}=\dot{\vec{v}}=\dsqdt{\vec{r}} = \cvec{\dsqdt{x_1}\\\dsqdt{x_2}\\\dsqdt{x_3}}
\end{align}

\subsection{Freiheitsgrade}
Die Anzahl der Freiheitsgrade eines Systems ist die Anzahl der nötigen Größen um die \textit{Lage} eines Systems zu beschreiben. Anders gesagt, es ist die Anzahl nötiger Größen um zu beschreiben wo alle relevanten Objekte im Raum liegen. Ein einzelner Massenpunkt hat also in einem dreidimensionalen Raum drei Freiheitsgrade. Ein System mit N Massenpunkten hat somit $3N$ Freiheitsgrade.

\subsection{Beispiele}
pass



\section{Koordinatensysteme}

Koordinatensysteme werden zur Beschreibung physikalischer Systeme verwendet. Allgemein bekannt ist das kartesische Koordinatensystem.
Eigentlich benötigt man nur drei verschiedene Arten von Koordinatensystemen, das kartesische Koordinatensystem,  das Zylinderkoordinatensystem und das Kugelkoordinatensystem. 

\begin{tikzpicture}
    % Draw axes
    \draw[->] (0,0) -- (3,0) node[right] {$x$}; % x-axis
    \draw[->] (0,0) -- (0,3) node[above] {$y$}; % y-axis
    \draw[->] (0,0) -- (-2,-2) node[below] {$z$}; %z-axis

    % Add grid (optional)
    %\draw[very thin, gray] (-3,-3) grid (3,3);

    % Add labels for the axes
    %\foreach \x in {-2,-1,1,2} % x-axis labels
    %    \draw (\x,0.1) -- (\x,-0.1) node[below] {\x};
    %\foreach \y in {-2,-1,1,2} % y-axis labels
    %    \draw (0.1,\y) -- (-0.1,\y) node[left] {\y};
        
    % Example point
    \filldraw[red] (1,2) circle (2pt) node[above] {$(x,y,z)$};
\end{tikzpicture}

\begin{tikzpicture}
    % Draw axes
    \draw[->] (0,0) -- (3,0) node[right] {$y$}; 
    \draw[->] (0,0) -- (0,3) node[above] {$z$}; 
    \draw[->] (0,0) -- (-2,-2) node[below] {$x$}; 

    % Draw spherical point
    \coordinate (O) at (0,0); % Origin
    \coordinate (P) at (1.5,1); % Point in spherical coordinates
    \coordinate (ProjP) at (1.5,-2); % Projection of P onto the x-y plane

    \draw[->, thick, red] (O) -- (1.4,0.93) node[above] {$\vec{ r }$}; % Radius vector
    \draw[dashed, gray] (P) -- (ProjP) node[below right] {}; % Projection line
    \draw[dashed, gray] (O) -- (ProjP) node[below] {}; % Projection vector in x-y plane

    % Add theta (polar angle)
    \draw[->, thick, blue] (0.5,0.33) arc[start angle=10, end angle=70, radius=0.7cm];
    \node[blue] at (0.2,0.5) {$\theta$};

    % Add phi (azimuthal angle)
    \draw[->, thick, green] (-0.5,-0.5) arc[start angle=-135, end angle=-40, radius=0.55cm];
    \node[green] at (-0.05,-0.38) {$\varphi$};

      % Draw labels
%    \node[red] at (1.5,1.5) {$(r, \theta, \phi)$};
    \filldraw[red] (P) circle (2pt); % Spherical point
\end{tikzpicture}

\begin{tikzpicture}
    % Draw axes
    \draw[->] (0,0) -- (3,0) node[right] {$y$}; 
    \draw[->] (0,0) -- (0,3) node[above] {$z$}; 
    \draw[->] (0,0) -- (-2,-2) node[below] {$x$}; 

    % Draw cylindrical point
    \coordinate (O) at (0,0); % Origin
    \coordinate (P) at (1.5,2); % Point in cylindrical coordinates
    \coordinate (ProjP) at (1.5,0); % Projection of P onto the x-y plane

    % Draw radial vector
    \draw[->, thick, red] (O) -- (1.5,0) node[below right] {$r$}; % Radial distance
    \draw[->, thick, gray] (1.5,0) -- (P) node[right] {$z$}; % Height

    % Add dashed line for projection
    \draw[dashed, gray] (O) -- (P) node[above right] {}; % Radius to point

    % Add phi (azimuthal angle)
    \draw[->, thick, blue] (0.7,0) arc[start angle=0, end angle=45, radius=0.7cm];
    \node[blue] at (0.8,0.3) {$\varphi$};

    % Draw labels
    \filldraw[red] (P) circle (2pt) node[above] {$(r, \varphi, z)$}; % Cylindrical point
\end{tikzpicture}
%\input{Inertialsysteme }
%\input{Galilei Transformation}
%\input{Newtonsche Axiome}
%\input{Energie}
%\input{Impuls}
%\input{Drehimpuls}
%\input{Definitionen}
%\input{Erhaltungssätze}
%\input{System von Massenpunkten}
%\input{Harmonischer Oszillator}
%\input{Zwei-Körper-Problem mit Zentralkraft}
%\input{Kepler}
%\input{Klassifizierung der Bahnen}
%\input{Rutherford-Streuung}

%\chapter{Mathematische Hilfsmittel}
%\input{Differential- und Integralrechnung}
%\input{Einfache Differentialgleichungen }
%\input{Potenzreihen}
%\input{Komplexe Zahlen}
%\input{Vektoren}
%\input{Gradient}
%\input{Linienintegral}
%\input{Delta Distribution}
\end{document}

%s/\<\w\+\>/\\input{\0}/g
